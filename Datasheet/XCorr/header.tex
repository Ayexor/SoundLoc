\documentclass[a4papper, 11pt]{article}

\title{XCorr IP core}
\author{Stefan Kull, Roy Seitz, Marco Zollinger}

\usepackage[T1]{fontenc}			%umlaute als eigene Zeichen
\usepackage[latin1]{inputenc}	%umlaute erkennen
\usepackage{lmodern}
\usepackage{amsmath} 					%mathematischen Textsatz.
\usepackage{amssymb}					%	-Erweiterte math. Sonderzeichen
\usepackage{dsfont}						%	-Mengen			
\usepackage[english]{babel}
\usepackage{textgreek}
\usepackage{siunitx}
\usepackage{caption}			%f�r \captionof
\usepackage{fancyhdr}					%f�r \pagestyle{fancy} -> kopf/fusszeile �ber Textk�rper hinaus
\usepackage{xfrac} %for fractions of style "1/2"
\usepackage{graphicx}
\usepackage{caption}


\usepackage[
	pdftitle={XCorr IP core},						%Titel des PDF Dokuments.
	pdfauthor={Stefan Kull, Roy Seitz, Marco Zollinger},									%Autor des PDF Dokuments.
	pdfsubject={IP block to effectively calculate the cross-correlation between three microphon signals in realtime},		%Thema des PDF Dokuments.
	pdfcreator={LaTeX with hyperref and KOMA-Script},		%Erzeuger des PDF Dokuments.
	pdfkeywords={xcorr, cross correlation, correlation, realtime, AXI4 Lite, AXI},						%auch f�r PDF Dokumente indexiert
	%	pdfpagemode=UseOutlines,								%Inhaltsverzeichnis anzeigen beim ffnen
	pdfdisplaydoctitle=true,								%Dokumenttitel statt Dateiname anzeigen.
	pdflang=en,												%Sprache des Dokuments.
	plainpages=false,
	hidelinks,												%keine Box um Links
	%	bookmarksopen.											%toc beim �ffnen expandiert
	pdfpagelabels
]{hyperref}

\usepackage[
	includeheadfoot,		%Kopf- und Fusszeilen verwenden
	headheight=15pt,		%H�he der Kopfzeile
	left=20mm,				%abstand von Seitenraendern
	right=20mm,				%
	top=20mm,
	bottom=20mm
	]{geometry}

%-------------------------------------------------
%Kopf-/Fusszeile
%-------------------------------------------------
\pagestyle{fancy}					%stil der kopf/fusszeilen 
\fancyhead{} 						% clear all header fields
\fancyhead[L]{SDM Decimator}
\fancyhead[R]{\leftmark }
\fancyfoot{} 						% clear all footer fields
\fancyfoot[C]{Page \thepage}
\fancypagestyle{plain}{				% plain redefinieren, damit wirklich alle seiten im gleichen stil sind (ausser titlepage)
	\pagestyle{fancy}
}
