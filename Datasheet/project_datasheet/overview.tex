\section{Overview}
Sound travels at a speed of approximately $c\approx \SI{340}{\meter\per\s}$.
Using multiple microphones at well known locations allows recalculating the direction from which the sound came.
Three microphones are placed in an equilateral triangle of 42 mm of side length.
A circle of 16 LEDs is placed on the same PCB to indicate the direction.
Because only the direction in $2D$ is of interest, three microphones are enough.
The microphone operate at approximately \SI{3.2}{\mega\hertz} using $\Sigma\Delta$-modulation.
They output their bitstream directly without any digital filtering or decimation.

The bitstream is filtered by a IP block, configurable by software via AXI4 Lite interface.
This block contains a CIC filter of configurable order and decimation rate with a differential Delay of $M=1$.
An additional IIR filter can be enabled to remove the DC component of the signals.
See Section \ref{sec::filter} for further information.

From the decimated Data, the cross-correlation is calculated to find the delay between the microphones.
This is done by another IP block, also configurable over AXI4 Lite.
See section \ref{sec::xcorr} for further information.

These delays are read by the CPU, which calculates the direction. 
In basically consists of a base transform from microphone in a cartesian coordinate system and calculating the inverse tangens of the coordinates.
This is described in section \ref{sec::software}.

The Angle is mapped to one of the 16 LED and then fed to another IP block that displays the direction.
It consists of a simple 16 bit shift register that illuminates the corresponding LED.
See section \ref{sec::display}.

