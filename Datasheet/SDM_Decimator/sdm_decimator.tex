\documentclass[a4papper, 11pt]{article}

\title{-- Digital Microelectronics --- SoundLoc -- \\ Localication of Soundsources by cross-correlating three $\Sigma\Delta$-microphon signals}
\author{Stefan Kull, Roy Seitz, Marco Zollinger}

\usepackage[T1]{fontenc}			%umlaute als eigene Zeichen
\usepackage[latin1]{inputenc}	%umlaute erkennen
\usepackage{lmodern}
\usepackage{amsmath} 					%mathematischen Textsatz.
\usepackage{amssymb}					%	-Erweiterte math. Sonderzeichen
\usepackage{dsfont}						%	-Mengen			
\usepackage[english]{babel}
\usepackage{textgreek}
\usepackage{siunitx}
\usepackage{graphicx}
\usepackage{caption}			%f�r \captionof

\usepackage[
	pdftitle={SoundLoc -- Digital Microelectronics Project},						%Titel des PDF Dokuments.
	pdfauthor={Stefan Kull, Roy Seitz, Marco Zollinger},									%Autor des PDF Dokuments.
	pdfsubject={Sound localication useing cross correlation of three microphones},		%Thema des PDF Dokuments.
	pdfcreator={LaTeX with hyperref and KOMA-Script},		%Erzeuger des PDF Dokuments.
	pdfkeywords={Cross correlatin, CIC, IP, AXI4\_Lite interface},						%auch f�r PDF Dokumente indexiert
	%	pdfpagemode=UseOutlines,								%Inhaltsverzeichnis anzeigen beim ffnen
	pdfdisplaydoctitle=true,								%Dokumenttitel statt Dateiname anzeigen.
	pdflang=en,												%Sprache des Dokuments.
	plainpages=false,
	hidelinks,												%keine Box um Links
	%	bookmarksopen.											%toc beim �ffnen expandiert
	pdfpagelabels
]{hyperref}

\usepackage[
%includeheadfoot,		%Kopf- und Fusszeilen verwenden
%headheight=15pt,		%H�he der Kopfzeile
left=30mm,				%abstand von Seitenraendern
right=30mm,				%
top=20mm,
bottom=20mm,
%			twoside
]{geometry}



\begin{document}
\maketitle
%\setcounter{tocdepth}{2}
%\tableofcontents

%\begin{table}[htbp]
%	\centering
%	\captionof{table}{Output signals of shift register controller}
%	\label{tbl::signals}
%	\begin{tabular}{|l|c|l|}
%		\hline 
%		A & B & C \\
%		\hline 
%	\end{tabular} 	
%\end{table}

\section{Overview}
\label{sec:overview}

This Document describes the functionality and usage of the SDM Decimator IP core.
It is supplied as a complete, ready to use IP core.

The project Soundloc contains three digital microphones that deliver a $\Sigma\Delta$-modulated (PDM) bitstream.
This IP generates the clock needed by the microphones and processes the received bitstreams.
It outputs filtered and decimated signed values and an interrupt to indicate when new values are available.
It is widely configurable by generic parameters at compile time as well as by software.
See section \ref{sec::parameters} and \ref{sec::driver} for detailed information.

The filter consists of a CIC filter, which is configurable up to 3rd order and a decimation up to its internal data width.
This is described in section \ref{sec::cic}.
An additional IIR DC blocker with a configurable pole can be added to remove a DC component. 
See section \ref{sec::iir} for detailed information.

Finally a post divider is implemented to scale the decimated data by powers of $\sfrac{1}{2}$.
This can be used to eliminate problems caused by overflow in the following cross-correlation that may occur with high order and decimation.
The post divider is controlled by software and described in section \ref{sec::post_divider}.

\section{Parameter description}
\label{sec::parameters}

The IP core can be configured at compile time by several parameters, listed in table \ref{tbl::parameters}.
\begin{table}[h]
	\centering
	\captionof{table}{Parameters for the SDM\_Decimator}
	\label{tbl::parameters}
	\begin{tabular}{|l|c|c|l|l|}
		\hline 
		Parameter & Default & Range & Type & Description \\ 
		\hline 
		D\_WIDTH & 16 & 1\ldots32 & integer & Internal data width \\
		\hline 
		D\_OUT\_WIDTH & 16 & 1\ldots18 & integer & Output data width \\
		\hline 
		DIVIDE & 40 & 4\ldots1024 & integer & Prescaler for bitstream clock \\
		\hline 
	\end{tabular} 
\end{table}

\newpage

\section{Register description}
\label{sec::registers}
Table \ref{tbl::register_space} describes the available registers and their functionality.
There are a total of five registers.
Each can be accessed directly by their address or more conveniently by the software driver described in section \ref{sec::driver}.
The individual bits in the status register are described in table  \ref{tbl::status_register}.

\begin{table}[h]
	\centering
	\captionof{table}{Register space}
	\label{tbl::register_space}
	\begin{tabular}{|l|c|c|l|l|}
		\hline 
		Name & Address offset & Mode & Description \\
		\hline 
		Status & 0x00 & R/W & Contains Status and control bits described in table \ref{tbl::status_register} \\
		\hline 
		Decimation & 0x04 & R/W & Contains the decimation of the CIC filter \\
		\hline 
		Value0 & 0x08 & R & Contains the actual, filtered value for microphone 2 \\
		\hline 
		Value1 & 0x0C & R & Contains the actual, filtered value for microphone 1 \\
		\hline 
		Value2 & 0x10 & R & Contains the actual, filtered value for microphone 3 \\
		\hline 
	\end{tabular} 
\end{table}

\begin{table}[h]
	\centering
	\captionof{table}{Status register}
	\label{tbl::status_register}
	\begin{tabular}{|l|c|l|}
		\hline 
		Bit		& Name 	& Description \\
		\hline 
		1:0		& order	& Binary coded order \\
				& 		& "00" deactivates filter. Outputs are set to 0 \\
		\hline 
		2 		& irq\_ena	& Enables the interrupt signal\\
		\hline 
		3		& iir\_ena	& Enables the IIR DC blocker \\
		\hline 
		7:4		& iir\_sr	& Sets the pole of the IIR DC blocker \\
		\hline 
		11:8	& post\_divide & Sets the post divider \\
		\hline 
	\end{tabular} 
\end{table}

\section{Bitstream settings}
\label{sec::bitstream}
The Bitstream clock is derived from the AXI Interface clock, prescaled by the parameter DIVIDE.
The microphones apply the new bits on falling edge and the Filter reads them on rising edge.
As the filter operates on signed values, the bitstream \{1,0\} is interpreted as \{+1, --1\}.

\section{CIC filter}
\label{sec::cic}

This section explains the functionality of the implemented CIC filter.
A CIC filter is fully characterized by its order, the differential delay and the decimation.
The resulting transfer function is shown in equation \ref{eq::cic_tf}.

\begin{align}
	H_{CIC}(z) &= \frac{\left(1-z^{-RM}\right)^N}{\left(1-z^{-1}\right)^N} = \left[\sum_{k=0}^{RM-1}z^{-k}\right]^N\label{eq::cic_tf}
\end{align}

The differential Delay is fixed to $M=1$.
The decimation $R$ and the order $N$ are set by software.
See section \ref{sec::driver} for further information.
The decimation is limited to the internal register width defined by parameter D\_WIDTH.
A value greater than $2^{D\_WIDTH} - 1$ results in undefined behaviour.
The order can be set to 0\ldots3.
\emph{Order 0 deactivates the filter and forces its outputs to 0.}

With high decimation and orders, overflow can occur.
The input is interpreted as 2bit signed value $\pm1$. 
Therefore, to avoid overflow the internal register width D\_WIDTH must be at least
\begin{align}
D\_WIDTH &\ge N\log_2{(RM)} + 1 \label{eq::cic_width}
\end{align}

\section{IIR DC blocker}
\label{sec::iir}
A simple IIR DC blocker is available to remove the DC component due to microphone offsets.
This may be needed as the correlation accumulates this DC component and may overflow.
The filter has the transfer function given in equation \ref{eq::iir_tf}.
The pole is set close to $z=1$ to obtain a low cutoff frequency.
The exact location is given in equation \ref{eq::iir_pole}.
$iir\_sr\in\left[0,15\right]$ is set by software.
\begin{align}
	H_{IIR} &= \frac{1-z^{-1}}{1-\left(1-2^{-iir\_sr}\right)z^{-1}} \label{eq::iir_tf}\\
	z_p &= \frac{2^{iir\_sr}-1}{2^{iir\_sr}} \label{eq::iir_pole}
\end{align}

\section{Post divider}
\label{sec::post_divider}

The post divider can divide the internally calculated values by powers of two.
The high gain from the CIC filter can therefore be scaled.
This is handy if the internal width is chosen greater than the output width.
The divider is implemented as a simple arithmetic shift right by post\_divider $\in\left[0\ldots15\right]$.
Thus setting post\_divider to 0 disables the divider.
The output value results in
\begin{align}
	out &= out_{internal}\cdot2^{-post\_divider}
\end{align}

\section{Driver}
\label{sec::driver}

The IP core comes along with a driver to allow easy access by software.
There are two categories of functions, high and low level.
The usage of high level functions is recommended.

\subsection{High level functions}
A typical use case with high level functions is as follows:
\begin{itemize}
	\item call SDM\_DECIM\_setStatus(baseaddr, mask) to initialize the IP core. \\
		This sets the content of the status register to mask. See section \ref{sec::registers} for further information.\\
		Typically: SDM\_DECIM\_setStatus(SDM\_DECIM\_BASEADDR, 0x307).
	\item Call SDM\_DECIM\_setDecimation(baseaddr, decimation). \\
		This sets the decimation. Typical values are about 30.
	\item If the raw values are of interest, call SDM\_DECIM\_getValue(baseaddr, mic).\\
		This returns a 32 bit signed integer that corresponds to the microphone mic. \\
		mic can be one of SDM\_DECIMATOR\_MICx, with $x\in\{1,2,3\}$.
\end{itemize}

\subsection{Low level functions}
The core can also be used by addressing the registers directly.
For this purpose two low-level functions are available.
These functions are implemented as macros and should only be used if really necessary.
The definitions from table \ref{tbl::software_defines} can be used to easily access these Registers.
Care must be taken to not confuse the microphones. 
On the PCB they are numbered from 1 to 3 but in hardware the numbering goes from 0 to 2.
The macros SDM\_DECIMATOR\_MICx are provided for mapping the indices.

\begin{itemize}
	\item SDM\_DECIM\_mWriteReg(BaseAddress, RegOffset, Data) \\
		Write data to the specified Register. 
	\item SDM\_DECIM\_mReadReg(BaseAddress, RegOffset) \\		
		Read the content of the specified register.
\end{itemize}

\subsection{Defined constants}

\begin{table}[h]
	\centering
	\captionof{table}{Software defined constants}
	\label{tbl::software_defines}
	\begin{tabular}{l|l}
		Name & Value \\
		\hline 
		SDM\_DECIMATOR\_S\_AXI\_REG0\_STATUS\_OFFSET & 0 \\
		\hline 
		SDM\_DECIMATOR\_S\_AXI\_REG1\_DECIMATION\_OFFSET & 4 \\
		\hline 
		SDM\_DECIMATOR\_S\_AXI\_REG2\_VALUE0\_OFFSET & 8 \\
		\hline 
		SDM\_DECIMATOR\_S\_AXI\_REG3\_VALUE1\_OFFSET & 12 \\
		\hline 
		SDM\_DECIMATOR\_S\_AXI\_REG4\_VALUE2\_OFFSET & 16 \\
		\hline 
		SDM\_DECIMATOR\_MIC1 & 1 \\
		\hline 
		SDM\_DECIMATOR\_MIC2 & 0 \\
		\hline 
		SDM\_DECIMATOR\_MIC3 & 2 \\
		\hline 
	\end{tabular} 
	
\end{table}


\section{Test and verification}
\label{sec::test}

The core can easily be simulated by using the AXI Trafic generator IP from Xilinx and feeding some value to the bitstream inputs.
Therefore no additional simulation files are available.
Software verification can be done by writing and reading data to the status and decimation registers.

\section{Compatibility and Licence}
The core is tested under Vivado version 2016.2 and on Artix7 and Zynq7 FPGA.
Since the core does not use any hardware specific resources, it should run on every other FPGA as well.
However, this is not tested and may require changes to the core hdl or driver files.
The core is supplied under no licence or copyright but is the intellectual property of the authors.

\end{document}