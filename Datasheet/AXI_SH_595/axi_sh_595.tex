\documentclass[a4papper, 11pt]{article}

\title{-- Digital Microelectronics --- SoundLoc -- \\ Localication of Soundsources by cross-correlating three $\Sigma\Delta$-microphon signals}
\author{Stefan Kull, Roy Seitz, Marco Zollinger}

\usepackage[T1]{fontenc}			%umlaute als eigene Zeichen
\usepackage[latin1]{inputenc}	%umlaute erkennen
\usepackage{lmodern}
\usepackage{amsmath} 					%mathematischen Textsatz.
\usepackage{amssymb}					%	-Erweiterte math. Sonderzeichen
\usepackage{dsfont}						%	-Mengen			
\usepackage[english]{babel}
\usepackage{textgreek}
\usepackage{siunitx}
\usepackage{graphicx}
\usepackage{caption}			%f�r \captionof

\usepackage[
	pdftitle={SoundLoc -- Digital Microelectronics Project},						%Titel des PDF Dokuments.
	pdfauthor={Stefan Kull, Roy Seitz, Marco Zollinger},									%Autor des PDF Dokuments.
	pdfsubject={Sound localication useing cross correlation of three microphones},		%Thema des PDF Dokuments.
	pdfcreator={LaTeX with hyperref and KOMA-Script},		%Erzeuger des PDF Dokuments.
	pdfkeywords={Cross correlatin, CIC, IP, AXI4\_Lite interface},						%auch f�r PDF Dokumente indexiert
	%	pdfpagemode=UseOutlines,								%Inhaltsverzeichnis anzeigen beim ffnen
	pdfdisplaydoctitle=true,								%Dokumenttitel statt Dateiname anzeigen.
	pdflang=en,												%Sprache des Dokuments.
	plainpages=false,
	hidelinks,												%keine Box um Links
	%	bookmarksopen.											%toc beim �ffnen expandiert
	pdfpagelabels
]{hyperref}

\usepackage[
%includeheadfoot,		%Kopf- und Fusszeilen verwenden
%headheight=15pt,		%H�he der Kopfzeile
left=30mm,				%abstand von Seitenraendern
right=30mm,				%
top=20mm,
bottom=20mm,
%			twoside
]{geometry}



\begin{document}
\maketitle
%\setcounter{tocdepth}{2}
%\tableofcontents

\section{Overview}
This Document describes functionality and usage of the AXI4 Lite 74xx595 Shift Register Controller.
It is supplied as a complete, ready to use IP block.

The controller allows easy access to 74xx595 compliant shift registers.
It contains an internal state machine to generate all needed signals, listed in table \ref{tbl::signals}.

The length of the Shift register is a generic and can be up to 32 bits, i.e. up to 4 shift registers in daisy-chain.
\emph{If the last shift register in daisy chain is not fully used, the unused output pins must be left floating, as their output is not defined.}

The data clock is derived from the AXI bus clock s\_axi\_aclk, which is divided by D\_DIVIDE.
See section \ref{sec::parameters} for detailed information.

\begin{table}[htbp]
	
	\captionof{table}{Output signals of shift register controller}
	\label{tbl::signals}
	\begin{tabular}{|l|c|l|}
		\hline 
		SH output signal & Reset / Idle value & Description \\ 
		\hline 
		rstn & 1 & Low active shift register master reset signal \\ 
		\hline 
		clk & 1 & Shift register data clock \\
		& & Data is shifted in on rising edge \\ 
		\hline 
		str & 1 & Shift register strobe signal \\
		 & & Data is load into output register on rising edge\\ 
		\hline 
		data & 0 & Shift register serial data in \\ 
		\hline 
		oen & 1 & Optional low active output enable \\
		& & If this signal is not used, it is set to constant 0 \\ 
		\hline 
	\end{tabular} 

\end{table}

\newpage

\section{Parameter description}
\label{sec::parameters}

The IP core can be configured at compile time by several parameters, listed in table \ref{tbl::parameters}.
\begin{table}[h]
	\captionof{table}{Parameters for the AXI 74xx595 controller}
	\label{tbl::parameters}
	\begin{tabular}{|l|c|c|l|l|}
		\hline 
		Parameter & Default & Range & Type & Description \\ 
		\hline 
		C\_SH\_DATA\_WIDTH & 8 & 1\ldots32 & integer & Length of the Shift register \\
		\hline 
		C\_USE\_OE\_N & true & \{true, false\} & boolean & Use output enable signal \\
		\hline 
		C\_CLOCK\_DECIMATION & 100 &  & integer & Prescaling factor for data clock \\
		\hline 
		C\_MSB\_FIRST & true & \{true, false\} & boolean & Shift Data out MSB first \\
		\hline 
	\end{tabular} 
\end{table}

\section{Register description}
\label{sec::registers}
Table \ref{tbl::register_space} describes the available registers and their functionality.
These registers can be accessed directly by their address or more conveniently by the software driver described in section \ref{sec::driver}.

\begin{table}[h]
	
	\captionof{table}{Register Space}
	\label{tbl::register_space}
	\begin{tabular}{|l|c|c|l|l|}
		\hline 
		Name & Address offset & Mode & Description \\
		\hline 
		Data & 0x0 & R/W & Contains LSB aligned data to write to the shift register \\
		\hline 
		Start/Ready & 0x4 & R/W & Write '1' start programming of shift register \\
		 & & & Read returns '1' if controller is ready to accept new data \\
		\hline 
		 Status & 0x8 & R/W & Bit 0: Enable controller \\
		 & & & Bit 1: Low active output enable (if oen is used) \\
		\hline 
	\end{tabular} 
	
\end{table}

\section{Driver}
\label{sec::driver}

The IP core comes along with a driver to allow easy access by software.
There are two categories of functions, high and low level.
The usage of high level functions is recommended.

\subsection{High level functions}
A typical use case with high level functions is as follows:
\begin{itemize}
	\item call AXI\_SH\_595\_init(baseaddr, mask) to initialize the controller. \\
		This sets the content of the status register to mask. \\
		Typically: AXI\_SH\_595\_init(AXI\_SH\_595\_BASEADDR, 0x1);
	\item Program shift register by calling AXI\_SH\_595\_programm\_sh(baseaddr, data) \\
		This writes data to the data register and starts the programming sequence.
	\item If consecutive calls of AXI\_SH\_595\_programm\_sh() are used, make sure the controller is ready to accept new commands by calling AXI\_SH\_595\_ready(baseaddr). \\
	This function returns 1 if the controller is ready and 0 if not.
\end{itemize}

\subsection{Low level functions}
The core can also be used by addressing the registers directly.
For this purpose two low-level functions are available.
These functions are implemented as macros and should only be used if really necessary.
The definitions from table \ref{tbl::software_defines} can be used to easily access these Registers.

\begin{itemize}
	\item AXI\_SH\_595\_mWriteReg(BaseAddress, RegOffset, Data) \\
		Writes data to the specified Register. 
	\item AXI\_SH\_595\_mReadReg(BaseAddress, RegOffset) \\		
		Read the content of the specified register.
\end{itemize}

\subsection{Defined constants}

\begin{table}[h]
	\centering
	\captionof{table}{Software defined constants}
	\label{tbl::software_defines}
	\begin{tabular}{l|l}
%		\hline 
		Name & Value \\
		\hline 
		AXI\_SH\_595\_DATA\_OFFSET & 0 \\
		\hline 
		AXI\_SH\_595\_START\_READY\_OFFSET & 4 \\
		\hline 
		AXI\_SH\_595\_STATUS\_OFFSET & 8 \\
%		\hline 
	\end{tabular} 
	
\end{table}


\section{Test and verification}
\label{sec::test}

The core can easily be simulated by using the AXI Trafic generator IP from Xilinx.
Therefore no additional simulation files are available.
Software verification can be done by writing and reading data to the data register, using low level function.

\section{Compatibility and License}
The core is tested under Vivado version 2016.2 and on Artix7 and Zynq7 FPGA.
Since the core does not use any hardware specific resources, it should run on every other FPGA as well.
However, this is not tested and may require changes to the core hdl or driver files.
The core is supplied under no license or copyright but is the intellectual property of the authors.

\end{document}